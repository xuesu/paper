\newpage
\begin{center}
	\heiti\zihao{3}\textbf{事件相关电位(ERPs)无线同步协议设计与系统实现} 
\end{center} 
\begin{center}
	\heiti\zihao{-3}\textbf{摘\quad 要}
\end{center} 
\vspace{2.5mm}
\songti\zihao{-4}  

事件相关电位(ERPs)是与刺激信号锁定的EEG脑电信号的叠加平均。与人脑认知活动相关的事件相关电位(ERPs)成分通过模式识别的方法应用于脑-机接口,可以实现人脑与计算机的直接通信,而不依赖于外围肌肉神经系统。

事件相关电位无线同步协议及其系统实现是为基于ERP的脑-机接口系统应用提供无线连接而设计的。该协议的核心思想是子钟同步,通过在原有系统中增加一个接收无线EEG脑电数据和有线刺激事件信号的无线同步触发器完成事件相关电位所需的同步。

协议通过一个有线连接的三次握手完成通信双方的连接和同步。系统实现基于ARM内核的STM32微控制器,并兼容了无线脑电放大器原有数据帧格式。

目前系统在没有做时钟偏差校准的情况下,子钟计时累积平均误差为177us/min。该系统(无线同步触发器)在每通道字节数固定的基础上,可以自适应无线脑电放大器对于采样频率和每采样点通道数的要求。一个数据包中的刺激事件信号个数在不超过32个时不会出现丢失的情况,能远远满足一般ERPs信号无线记录的要求。每个发送到计算机的数据包的大小限制为$1024Bytes$。最后,无线同步触发器能最高接收$10KHz$采样速率的数据并提供相应精度的刺激事件信号。

\vspace{3mm}
\zihao{-4}\heiti\textbf{关键字}\quad \songti 事件相关电位 \quad 无线 \quad 脑机接口

