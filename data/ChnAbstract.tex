\newpage
\begin{center}
	\heiti\zihao{3}\textbf{基于深度神经网络的文本情感分类的研究与实现} 
\end{center} 
\begin{center}
	\heiti\zihao{-3}\textbf{摘\quad 要}
\end{center} 
\vspace{2.5mm}
\songti\zihao{-4}  

情感分析(Sentiment Analysis, SA),也称为观点挖掘(Opinion/Review Mining),指分析说话者所传达信息中隐含的情况状态,态度,意见等,在舆情分析,自动决策支持,广告推荐等很多方面都具有很大价值。文本情感分类是情感分析任务中的一种,主要识别给定文本的情感极性,由于自然语言文本的词汇丰富,又具有语义多样性,而且往往具有复杂的语法结构及不规范的语用现象如冗余,重复等,目前文本情感分类依然是一个具有挑战性的任务。本文主要研究如何使用深度学习完成语句级的文本情感分类任务,同时也实现了用于完成情感分类任务的知识模型以进行研究和对比。


本文的主要工作是构建了三个模型,首先是基于Stanford语法树的知识模型,其次,由于过于复杂的句式反而可能给神经网络带来干扰,本文提出了由C\&W模型改进,抛弃了全局语法特征的短语级卷积模型(Convolutional Neural Network on Phrase Level, CNNPL)。最后,由于CNNPL模型不能处理不定长的语句,也不能利用自然语言的序列性质,而单纯的循环神经网络如流行的LSTM模型不能提取文本的短语特征,因此本文结合CNNPL模型与LSTM模型,提出了短语级带有LSTM细胞的卷积模型(Convolutional Neural Network with LSTM on Phrase Level, CNNLSTMPL)。其中CNNPL模型和CNNLSTMPL模型均在NLPCC 2014的使用深度学习进行情感分析的任务集(SCDL)上取得了接近或略超过第一名的性能。


与此同时,本文提出了一套神经网络处理情感分类任务的流程,并对该流程中重要的部分进行参数分析。最后,本文采用Restful API框架实现了一套基于网页的展示系统。

\vspace{3mm}
\zihao{-4}\heiti\textbf{关键字}\quad \songti 深度学习 \quad 情感分析 \quad 长短期记忆模型 \quad 卷积神经网络

