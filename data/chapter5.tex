\chapter{讨\quad 论}

至此,关于系统实现及结论就全部阐述了。在实际系统设计和实现过程中,遇到很多困难,也走了一些弯路,遇到一些问题没有克服,暂时跳过去的也有。本章的主要目的就是针对毕设过程中发生的这些问题给予总结,以方便以后进一步的工作。

\section{蓝牙等间隔传输实现无线同步方案}

	这是本毕设开始时最早提出的系统实现方案,方案原理阐述如下:图~\ref{WiredBCI}展示的是ERPs有线同步的系统框图,在这里不需要如无线传输所需协议的原因在于存在如下假设,且这个假设在该问题下近似是真的。

\begin{center}
\textbf{假设:有线信道传输数据的延迟为0}
\end{center}

因此从头皮EEG脑电记录到的数据发送到计算机(或者到放大器)的时间与同时产生的Trigger刺激信号到达的时间之间没有延迟。但是在无线传输中,蓝牙无线信道的传输延迟不能被忽略。但是否可以逼近?

	这就是蓝牙等间隔传输方案的基本思想,其目标在于实现:即使是无线信道的情况下,脑电放大器通过无线发送的数据也能以1ms一次的等间隔被无线同步触发器接收到,无线同步触发器在接收到蓝牙数据的同时去读取是否有刺激信号,并标记。

	该方案完全依赖于无线模块的性能,即如果蓝牙模块要达到如下几点指标:
\begin{enumerate}
\item	蓝牙模块能在1ms间隔下传输需要数量的数据
\item	每次数据到达接收端蓝牙并通过接收端蓝牙串口给无线同步触发器的时间间隔也必须在1ms左右
\item	接收端蓝牙通过串口给的数据不能自己再打包,而必须是发送端给什么样大小的数据,接收端就给出什么样大小的数据包。
该协议的可能性在于:对于蓝牙无线传输的系统误差,即平均延迟可以忍受,但对于随机误差必须要小,当时期望的指标在$\pm 1ms$的量级。
\end{enumerate}

	然而BTM0704蓝牙模块无法达到该指标,相关结论见 ~\ref{sec:bluetooResult}节。一次实际的数据发送情况测试见图~\ref{10ms115200SendRec}:可以明显看到上面的接收波形,各脉冲之间的时间间隔不相等。

\begin{figure}[!hbp]
\begin{center}
\includegraphics[scale=0.5]{graphic/expData/10ms115200.jpg}
\caption{115200蓝牙波特率10ms传输间隔发送数据,接收间隔监视 \label{10ms115200SendRec}}
\end{center}
\end{figure}

在与设备制造商联系以及考虑其他途径获得更高精度的蓝牙模块未果的情况下,放弃该方案。

\textbf{总结:}相比于子钟对准方案,蓝牙等间隔传输的方案完全依赖于蓝牙模块的无线传输方差,而无线信道的误差绝对存在,从蓝牙模块的有关参数~\footnote{参考重庆金瓯BTM0704蓝牙模块技术手册,或参考\ref{subsec:buletooth}节}可知,该蓝牙模块在设计时就通过前向纠错和反馈重传的方式来保证数据传输的正确性,而这些操作本身就是造成延时的原因,从另一个侧面也反映了无线信道相比有线信道条件要更恶劣一些。

而子钟对准的方案的精度除去微控制器,完全取决于晶振的精度,在这里是50ppm,对于带温度补偿的晶振可以做到10ppm以下,大约是$10^{-6}$的精度。而且从\ref{sec:SourceOscillator}节的结论来看,其误差远小于蓝牙无线传输,且该误差为累积误差,具有一定的可纠正性。

\section{ERPs对于误差精度的要求}

到底何种精度的误差是事件相关电位ERPs所能忍受的?这主要取决于事件相关电位中的有效成分,或者说是科研人员关心的成分有关。如图~\ref{ERPComponents}所示ERPs研究中主要关心的信号成分主要是在刺激发生以后100ms到800ms出现的一些峰值信号,这些信号相对于刺激时刻出现的时间位置称之为潜伏期,这决定了误差信号的第一个精度范围,误差不能达到100ms量级。

其次,脑电的模式识别中需要信号的峰值比较明显才能做出判断,但有些成分其峰值出现时很尖锐,信号幅度大于$50 \%$的时间小于10ms,如果此时刺激时刻的定位具有$\pm 10ms$的偏差,那么就可能在多试次的叠加平均中把信号展宽。

综合以上两条,可以对ERPs信号对于误差精度的需求有一个定性的概念。

\section{硬件系统设计问题}

\subsection{PCB}
	PCB采用8mil线间距,最小过孔内径15mil,外径28mil双层板工艺,大部分元件都采用了贴片封装。由于是第一次正式画PCB,对于设计中需要注意的几个问题,以及需要改进的地方阐述如下:
\begin{itemize}
\item	USB端口传输的是差分信号,也即通过两条信号线的信号差值传输信息,必须保证USB的两条线的布线长度一样,防止线长不一致导致的相移造成差错信号。
\item	并口有母口和公口之分,并且两者的封装镜像对称。本设计中没有注意到这个问题,使用了母口的封装,因此实际焊接公口时是在PCB底层插入,在顶层焊接的,如图~\ref{hardwarePCB}所示。
\end{itemize}

\subsection{并口5V耐压}

	在考虑使用哪几个IO口作为并口输入时,由于首先考虑到的是如何在软件读取时的便利性,比如用PC0到PC7端口作为并口数据读取,只需要一个指令就可以同时读取八个IO口的数据而忽略了并口是用0V表示低电平而用5V表示高电平。已知STM32的系统供电为3.3V,因此其大部分IO管脚都是3.3V耐压的,其由部分管脚被设计成5V耐压。在设计时我并没有注意到这点,目前使用的并口从PC0到PC7中只有PC6和PC7具有5V耐压。因此在提供刺激事件信号时,目前该PCB最好使用0x40或者0x80作为刺激事件码。


\section{软件编程问题}

\subsection{1Kbytes发送限制}

目前还不清楚1Kbytes传输限制的原因。在无线同步触发器发送数据返回计算机时,检测到如下问题:当发送的一个数据包(返回计算机数据的数据包既有EEG脑电数据同时还有刺激事件信号)的大小超过1024bytes时,从无线同步触发器发送到计算机的数据会出现不正常的情况。

测试时,每通道4字节,每个采样点两通道,外加一个刺激事件信号的通道就是每采样点三通道,如果每次发送85个采样点的数据,就是$1020bytes$,外加帧头的五个字节是1025字节,该情况下没有错误。但是当发送86个采样点数据时,也即共$1032bytes + 5 bytes = 1037 bytes$时,从计算机上接收到的数据格式不符合要求。以接收到的两组数据为例(以下表示均为16进制),帧头是0xF0 0x0F,帧号为0x06,0x24代表每采样点两通道,每通道4字节,0x0A表示10进制的十,即该数据包有十个采样点的数据。该两组的每通道有效数据分别都为11 11 11 11和DD DD DD DD。可以看见,从计算机上接收到的如下数据在每两个通道八个字节以后空出了四个字节(16进制一个数字代表了4bit)作为刺激事件码存放通道。在这里我做了一个顺序的标记,因此可以看到每个刺激事件通道的第二个字节从01递增到0A刚好为10个刺激通道。此时的一个发送数据包数据总量为:$5 + 3 \times 4 \times 10 = 125 bytes$远小于1024bytes。

{\zihao{5} \underline{\bfseries F0 0F 06 24 0A} ~\underline{11 11 11 11} ~\underline{11 11 11 11} 00 01 00 00 ~\underline{11 11 11 11} 11 11 11 11 00 02 00 00 11 11 11 11 11 11 11 11 00 03 00 00 11 11 11 11 11 11 11 11 00 04 00 00 11 11 11 11 11 11 11 11 00 05 00 00 11 11 11 11 11 11 11 11 00 06 00 00 11 11 11 11 11 11 11 11 00 07 00 00 11 11 11 11 11 11 11 11 00 08 00 00 11 11 11 11 11 11 11 11 00 09 00 00 11 11 11 11 11 11 11 11 00 0A 00 00 F0 0F 07 24 0A DD DD DD DD DD DD DD DD 00 01 00 00 DD DD DD DD DD DD DD DD 00 02 00 00 DD DD DD DD DD DD DD DD 00 03 00 00 DD DD DD DD DD DD DD DD 00 04 00 00 DD DD DD DD DD DD DD DD 00 05 00 00 DD DD DD DD DD DD DD DD 00 06 00 00 DD DD DD DD DD DD DD DD 00 07 00 00 DD DD DD DD DD DD DD DD 00 08 00 00 DD DD DD DD DD DD DD DD 00 09 00 00 DD DD DD DD DD DD DD DD 00 0A 00 00}

不改变任何接收程序,只是增大每个包的采样点树立,使发送缓冲区数组字节数大于1024bytes,如下面一组数据帧头所示:每采样点2通道,每通道4字节,共0x64,即100个采样点,加上每个采样通道之后的刺激事件信号通道,总数据字节数为$4 \times 3 \times 100 + 5 = 1205 bytes$超过1024字节,就出现了问题。
\begin{enumerate}
\item 总的数据字节是正确的,从帧头以及最后一个通道的刺激事件码通道顺序号0x64可以得出确实是发送了100个通道的数据字节数;
\item 帧头数据正确;
\item 从第一个通道开始的数据部符合规律,没有如上面一组数据在经过八个字节以后空出一个通道的刺激事件信号通道;
\item 数据段起始的几个00刚好是每采样点1个通道的情况下擦除刺激事件码的位置,每次数据发送前都需要先清除上一次数据操作留下来的刺激事件码;但从帧头来看,应该是每采样点两个通道,擦除位置在第三个通道的第一个字节才是正确的;
\item 从第16个通道(0x10)开始数据格式又恢复了正常,但是最后一段的连续0表明数据丢失,而这些丢失的数据很有可能就是在数据段前面没有做移位的数据。
\end{enumerate}

由于测试该段数的代码与前面一次测试的数据是完全相同的,无线同步触发器通过自动接收无线脑电放大器给出的参数修改自身的接收参数,而且发送接收缓冲区的初始化大小都大于2048bytes。再加上后来测试发现的数据格式刚好在数据量大于1024bytes之后出错,认定该问题应该跟系统的某个1024bytes大小的区域有关。

曾经怀疑该区域与中断有关,但是在将该段代码不使用中断来完成以后(完全用轮询),问题仍然存在。

{\zihao{5} \underline{\bfseries F0 0F 2A 24 64} ~\underline{11 11 11 11} ~\underline{00 11 11 11} 11 11 11 11 ~\underline{11 11 11 11} 00 11 11 11 11 11 11 11 11 11 11 11 00 11 11 11 11 11 11 11 11 11 11 11 00 11 11 11 11 11 11 11 11 11 11 11 00 11 11 11 11 11 11 11 11 11 11 11 00 11 11 11 11 11 11 11 11 11 11 11 00 11 11 11 11 11 11 11 11 11 11 11 00 11 11 11 11 11 11 11 11 11 11 11 00 11 11 11 11 11 11 11 11 11 11 11 00 11 11 11 11 11 11 11 11 11 11 11 00 11 11 11 11 11 11 11 11 11 11 11 00 11 11 11 11 11 11 11 11 11 11 11 00 11 11 11 11 11 11 11 11 11 11 11 00 11 11 11 11 11 11 11 11 11 11 11 00 11 11 11 00 0F 00 00 11 11 11 11 11 11 11 11 00 10 00 00 11 11 11 11 11 11 11 11 00 11 00 00 11 11 11 11 11 11 11 11 00 12 00 00 11 11 11 11 11 11 11 11 00 13 00 00 11 11 11 11 11 11 11 11 00 14 00 00 11 11 11 11 11 11 11 11 00 15 00 00 11 11 11 11 11 11 11 11 00 16 00 00 11 11 11 11 11 11 11 11 00 17 00 00 11 11 11 11 11 11 11 11 00 18 00 00 11 11 11 11 11 11 11 11 00 19 00 00 11 11 11 11 11 11 11 11 00 1A 00 00 11 11 11 11 11 11 11 11 00 1B 00 00 11 11 11 11 11 11 11 11 00 1C 00 00 11 11 11 11 11 11 11 11 00 1D 00 00 11 11 11 11 11 11 11 11 00 1E 00 00 11 11 11 11 11 11 11 11 00 1F 00 00 11 11 11 11 11 11 11 11 00 20 00 00 11 11 11 11 11 11 11 11 00 21 00 00 11 11 11 11 11 11 11 11 00 22 00 00 11 11 11 11 11 11 11 11 00 23 00 00 11 11 11 11 11 11 11 11 00 24 00 00 11 11 11 11 11 11 11 11 00 25 00 00 11 11 11 11 11 11 11 11 00 26 00 00 11 11 11 11 11 11 11 11 00 27 00 00 11 11 11 11 11 11 11 11 00 28 00 00 11 11 11 11 11 11 11 11 00 29 00 00 11 11 11 11 11 11 11 11 00 2A 00 00 11 11 11 11 11 11 11 11 00 2B 00 00 11 11 11 11 11 11 11 11 00 2C 00 00 11 11 11 11 11 11 11 11 00 2D 00 00 11 11 11 11 11 11 11 11 00 2E 00 00 11 11 11 11 11 11 11 11 00 2F 00 00 11 11 11 11 11 11 11 11 00 30 00 00 11 11 11 11 11 11 11 11 00 31 00 00 11 11 11 11 11 11 11 11 00 32 00 00 11 11 11 11 11 11 11 11 00 33 00 00 11 11 11 11 11 11 11 11 00 34 00 00 11 11 11 11 11 11 11 11 00 35 00 00 11 11 11 11 11 11 11 11 00 36 00 00 11 11 11 11 11 11 11 11 00 37 00 00 11 11 11 11 11 11 11 11 00 38 00 00 11 11 11 11 11 11 11 11 00 39 00 00 11 11 11 11 11 11 11 11 00 3A 00 00 11 11 11 11 11 11 11 11 00 3B 00 00 11 11 11 11 11 11 11 11 00 3C 00 00 11 11 11 11 11 11 11 11 00 3D 00 00 11 11 11 11 11 11 11 11 00 3E 00 00 11 11 11 11 11 11 11 11 00 3F 00 00 11 11 11 11 11 11 11 11 00 40 00 00 11 11 11 11 11 11 11 11 00 41 00 00 11 11 11 11 11 11 11 11 00 42 00 00 11 11 11 11 11 11 11 11 00 43 00 00 11 11 11 11 11 11 11 11 00 44 00 00 11 11 11 11 11 11 11 11 00 45 00 00 11 11 11 11 11 11 11 11 00 46 00 00 11 11 11 11 11 11 11 11 00 47 00 00 11 11 11 11 11 11 11 11 00 48 00 00 11 11 11 11 11 11 11 11 00 49 00 00 11 11 11 11 11 11 11 11 00 4A 00 00 11 11 11 11 11 11 11 11 00 4B 00 00 11 11 11 11 11 11 11 11 00 4C 00 00 11 11 11 11 11 11 11 11 00 4D 00 00 11 11 11 11 11 11 11 11 00 4E 00 00 11 11 11 11 11 11 11 11 00 4F 00 00 11 11 11 11 11 11 11 11 00 50 00 00 11 11 11 11 11 11 11 11 00 51 00 00 11 11 11 11 11 11 11 11 00 52 00 00 11 11 11 11 11 11 11 11 00 53 00 00 11 11 11 11 11 11 11 11 00 54 00 00 11 11 11 11 11 11 11 11 00 55 00 00 11 11 11 11 00 00 00 00 00 56 00 00 00 00 00 00 00 00 00 00 00 57 00 00 00 00 00 00 00 00 00 00 00 58 00 00 00 00 00 00 00 00 00 00 00 59 00 00 00 00 00 00 00 00 00 00 00 5A 00 00 00 00 00 00 00 00 00 00 00 5B 00 00 00 00 00 00 00 00 00 00 00 5C 00 00 00 00 00 00 00 00 00 00 00 5D 00 00 00 00 00 00 00 00 00 00 00 5E 00 00 00 00 00 00 00 00 00 00 00 5F 00 00 00 00 00 00 00 00 00 00 00 60 00 00 00 00 00 00 00 00 00 00 00 61 00 00 00 00 00 00 00 00 00 00 00 62 00 00 00 00 00 00 00 00 00 00 00 63 00 00 00 00 00 00 00 00 00 00 00 64 00 00  }


在没有解决这个问题的情况下,无线同步触发器会因为这个软件问题,每次的返回数据包字节数不大于1024字节。

\subsection{JTAG端口复用}
	原先的一导无线脑电放大器并没有多余的管脚,而要实现无线同步触发器的握手过程,需要两个普通IO口分别配置为输入和输出。JTAG端口原先作为程序调试端口存在,在一般情况下都不占用作为普通IO,因为一旦使能为普通IO,下一次想烧录程序时下载工具就会报无法找到设备的错误。

	为了复用JTAG端口,可以在每次系统上电之后先延迟3s,然后再初始化JTAG端口的管脚为普通IO口,这样就可以在每次上电时对程序进行重新烧录,又可以使用JTAG的引脚为普通IO口。唯一缺点是增加了系统的初始化时间。

	设置JTAG端口为普通IO的STM32固件库代码如下:
\begin{center}
\verb|	GPIO_PinRemapConfig(GPIO_Remap_SWJ_Disable, ENABLE);|
\end{center}

该代码关闭了JTAG端口的所有管脚功能,根据固件库的描述,还可以设置成只关闭部分管脚~\footnote{相关请参考STM32 Reference Manual RM0008}

	另外还需要使能STM32管脚第二功能的时钟,代码如下:
\begin{center}
\verb|	RCC_APB2PeriphClockCmd(RCC_APB2Periph_AFIO, ENABLE);|
\end{center}

	如果不小心把JTAG端口设置为普通IO而无法下载程序,可以通过上拉BOOT/BOOT1引脚来烧录新的程序。这是由于系统在上电或RESET复位以后和进入调试模式以前,调试器需要一些时间初始化JTAG端口,然后才能"冻结"微控制器。如果程序中"借用JTAG引脚作为IO口的指令早于调试器初始化",则调试器不能接管微控制器,就会产生JTAG连接失败。把BOOT/BOOT1上拉意味着改为内部SRAM启动,这样系统复位或上电以后就不能执行借用引脚指令,调试器就可以正常连接芯片中的JTAG模块并成功接管微控制器,以正常下载程序。

\subsection{数据包超前错误}
	参考图~\ref{CorrespondingFIDCounter},该模型比较容易犯的一个错误是发送的包的FrameID超前于该数据包应该出现的时间段而导致刺激事件码丢失。比如FrameID为0的包按照实际情况是在{\textcircled{\small{2}}}后完成采样再发送,到达无线同步触发器的时间在时间{\textcircled{\small{2}}}以后,图中假设为{\textcircled{\small{3}}}时刻。此时无线同步触发器(图~\ref{CorrespondingFIDCounter} 中A)记录的是时间段$\delta t$内可能出现的Trigger信号,而脑电放大器(图~\ref{CorrespondingFIDCounter}中B)正在采样的是{\textcircled{\small{2}}}到{\textcircled{\small{4}}}之间的数据,无线同步触发器将正确的把{\textcircled{\small{1}}}到{\textcircled{\small{2}}}的时间段内的Trigger标记到FrameID为0的数据包中,但是如果发送时序出错,{\textcircled{\small{3}}}时刻收到的是是FrameID为1的数据包,那么无线同步触发器记录到的{\textcircled{\small{1}}}到{\textcircled{\small{2}}}的时间段的Trigger就不会被标记,最后被新的Trigger覆盖,以此类推,很容易得出结论就是后面的Trigger也会丢失。
	如上所述,FrameID为1的包在无线同步触发器认为其应该出现的时间之前出现导致了上面的错误,对应实际情况就是{\textcircled{\small{2}}}到{\textcircled{\small{4}}}时间段的采样点数据都到了,但是实际时间还停留在{\textcircled{\small{2}}}到{\textcircled{\small{4}}}之间,这不符合实际,出错原因是脑电放大器并没有遵守无线同步触发器的数据模型。在另一种情况中,数据包符合实际情况但晚于其采样完成时间之后到达就不会导致Trigger丢失的问题,在延后数据包出现的情况下,Trigger丢失的原因是由于系统所能记录的旧Trigger被新的Trigger覆盖所导致的,所以增加无线同步触发器所能保存的采样点数量就能避免旧Trigger过早的被新Trigger覆盖,而该系统设计中,目前最大能满足一个数据包标记32个刺激事件信号,对于通常1s或800ms的一个刺激事件而言,一个数据包的采样时间内出现两个刺激事件的概率=0

\subsection{使用系统本身有源晶振做误差补偿的尝试}

	在物理实验中要测量一个仪器的误差,需要比他精度至少高两个数量级的仪器。在本设计中的时钟偏差如果需要纠正,也应该采用这种措施,通过示波器得出的线下数据非常有说服力,但是如果能实现系统实时的数据纠正,或者说针对某个具体有源晶振的数据纠正,将更具有个体适应性。由于在每次DMA数据接收完成后,无线同步触发器都需要进行乒乓数组的交替,本毕设中尝试使用每次完成DMA操作的时间来进行误差纠正。

	测试使用无线同步触发器和一导的无线脑电放大器,都带有50ppm的有源晶振,使用相同的主频36MHz。每次无线脑电放大器给1s长10个采样点的数据,无线同步触发器记录了60次数据接收以后给出60s的时间差,按照正常情况,该值应该为600000上下波动,可以预见的是该值还包含了无线传输的传输波动。结果如图~\ref{WirelessTriggerCalDelay} 所示,结合第四章 \ref{sec:bluetooResult}节的相关定性结论,该图所示的结果更倾向于表现无线传输延迟的波动情况,由于没有一致性的规律,所以的误差纠正方案不可行,采用线下数据对无线同步触发器的同步计数值进行修正。


\begin{figure}[!hbp]
\begin{center}
\includegraphics[width=\textwidth]{graphic/equalSendBlueTooth.eps}
\caption{使用无线同步触发器晶振读取脑电放大器传输间隔图 \label{WirelessTriggerCalDelay}}
[对所有值已减去600,000,并把纵轴单位换算为ms]
\end{center}
\end{figure}

