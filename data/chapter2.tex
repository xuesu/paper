\chapter{数据获取与预处理}
本章将从系统使用的数据集,情感词典和预处理流程三方面进行介绍。
\section{数据集}
为了评估数据模型的准确性以及训练模型以便展示等目的,本文针对语句级和情感级两个任务获取了以下数据集。
\subsection{语句级}
\begin{table}  
\caption{语句级数据集}  
\begin{center}  
\begin{tabu} to 0.8\textwidth{X[c]|X[3]} 
\hline
名称 & 手机评论数据集\\
\hline
数据总数 & 2700条\\
语言 & 中文\\
比例 & 正向:负向=1700:1000\\
词数最大值 & 300词数量级\\
90\%词数 & 50词数量级\\
情感分类准确度 & 较为正确\\
语句特点 & 语句较为规范,感情较为强烈\\
获取方式 & 实验室内部资源\\
主要用途 & 初期评估模型可行性及模型展示\\
\hline
名称 & NLPCC 2014 SCDL数据集\\
\hline
数据总数 & 中文12500条,英文12485条,共24985条\\
官方测试集数据总数 & 中文2500条,英文2500条\\
语言 & 中文,英文\\
比例 & 中文 正向:负向=1700:1000\\
测试集比例 & 中文 正向:负向=1250:1250 英文 正向:负向=1250:1250\\
词数最大值 & 1000词数量级\\
90\%词数 & 200词数量级\\
情感分类准确度 & 不够准确\\
语句特点 & 语句较为口语化,有广告等中性语句\\
获取方式 & \url{http://tcci.ccf.org.cn/conference/2014/pages/page04_sam.html}(Accessed at: 5/21/2017)\\
主要用途 & 评估模型性能\\
\hline
名称 & 携程数据集\\
\hline
数据总数 & 250463条\\
语言 & 中文\\
比例 & 正向:负向=193138:57325\\
词数最大值 & 1750词数量级\\
90\%词数 & 250词数量级\\
情感分类准确度 & 不够准确\\
语句特点 & 语句较为口语化\\
获取方式 & 通过selenium模拟浏览器爬取到388067条带评分的携程酒店评论,其中评分小于4.0的定为负向,评分等于5的定为正向\\
主要用途 & 模型展示及标注实体数据集\\
\hline
\end{tabu}
\end{center}
\end{table}

\subsection{实体级}

\begin{table}  
\caption{实体级数据集}  
\begin{center}  
\begin{tabu} to 0.8\textwidth{X[c]|X[3]} 
\hline
名称 & SemEval2014Task4数据集\\
\hline
数据总数 & Laptop子数据集为3045条,Restaurants子数据集为3041条\\
语言 & 英文\\
词数最大值 & 80词数量级\\
90\%词数 & 50词数量级\\
情感分类准确度 & 较为正确\\
语句特点 & 语句较为规范,感情较为强烈\\
获取方式 & \url{http://alt.qcri.org/semeval2014/task4/}(Accessed at: 5/21/2017),取其中已标注的训练数据部分\\
主要用途 & 评估模型可行性及模型展示\\
\hline
名称 & 携程实体数据集\\
\hline
数据总数 & 6947条\\
语言 & 中文\\
词数最大值 & 100词数量级\\
90\%词数 & 20词数量级\\
情感分类准确度 & 较为正确\\
语句特点 & 语句较为规范,感情较为强烈\\
获取方式 & 对携程数据集进行过滤,取长度为10-100词之间,名词类词语限制为携程数据集上频率最高的20个词的语句进行手工标注实体和实体极性得到\\
主要用途 & 评估模型可行性及模型展示\\
\hline
\end{tabu}  
\end{center}  
\end{table} 
