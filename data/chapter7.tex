\chapter{结束语}\thispagestyle{fancy}
本文研究如何使用深度神经网络对文本进行情感分类,实现了基于卷积神经网络的NOLSTM模型,和结合卷积神经网络与LSTM的CLSTM模型,二者都在NLPCC2014SCDL数据集上取得了接近第一名的结果。与此同时,本文实现了基于知识和情感词典的知识模型,用于与神经网络模型相比较。\par
本文实现了一套可以同时应用于中英双语的情感分类器,在实现上,该分类器通过面向对象编程的思想,能够通过只加载配置生成保存不同的模型,便于研究对比。同时,本文基于Restful API实现了一套用于展示的网页。\par
项目从2016年10月末开始准备,一直到2017年6月初结束,前期主要实现语义级的情感分析任务,完成模型并调参,后期则致力于实体级的情感分类任务,但考虑问题不够完善,配套的两个模型没有达到足以写入论文的精度。\par
项目还可以改进和继续努力的地方有:\par
1. 寻找足够稳定,精度足够,结构有拓展性的局部卷积和池化方案。\par
2. 知识模型应考虑更加细致的规则,以提高精度。同时,应当考虑更加详细的情感分级,或者基于不同主题赋予单词不同的情感权重。